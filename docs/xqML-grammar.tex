%% Author: Tahir Hashmi
%% 
%% Copyright (C) 2002 2003, Xqueeze Developers
%% 
%% This file is part of Xqueeze Documentation
%% 
%% Permission is granted to copy, distribute and/or modify this
%% document under the terms of the GNU Free Documentation License,
%% Version 1.2 or any later version published by the Free Software
%% Foundation; with no Invariant Sections, no Front-Cover Texts, and no
%% Back-Cover Texts. A copy of the license is included in LaTeX source
%% format in the file entitled ``fdl.tex''.

\subsubsection{Terminals}
\begin{itemize}
\item \textit{Figures enclosed within braces ($\{\}$)} are hex codes for the value
  of an xqML Symbol that should occur within.
\item \textit{Rev} is an octet to be interpreted as an unsigned integer.
\item \textit{xqA} is the inline Xqueeze Association with prolog
  (see \S{} \ref{section:xqa-fmt}).
\item \textit{ELSymbol}, \textit{ATSymbol}, \textit{APSymbol},
  \textit{VASymbol} and \textit{ENSymbol} are all xqML symbols derived
  from an Xqueeze Association to represent the vocabulary of an XML
  document type.
\item \textit{NSSymbol} is an xqML Symbol of the type ``namespace prefix''
  (has a document-specific value).
\item \textit{ElementsToClose} is an octet to be interpreted as an unsigned
  integer.
\item \textit{VUint} is a special type of xqML Symbols that represents
  Variable-length Unsigned integers (see \S{} \ref{section:xqMLSymbols}).
\end{itemize}

\subsubsection{Productions}
\begin{enumerate}
\item \label{pr:document} \(document\;::=\;prolog\;element\;PI* \)

Every xqML document must match the above production. Thus,
\textit{document} is the starting symbol.

\item \label{pr:prolog} \(prolog\;::=\;[^\wedge\{0x1E\}]\!*\;xqMLDecl\;PI\!*\;(doctypedecl\;PI*)? \)

The \textit{prolog} of an xqML document can contain anything upto the
first occurrence of xqML Symbol \{0x1E\}.

\item \label{pr:xqMLDecl} \(xqMLDecl\;::=\;'\{0x1E\}'\;Fmt\;Rev\;Char* \)

Every xqML document must declare what it is (xqML), its binary format
and the version of its encoding. {\it Rev} is an octet that represents
the revision number of the xqML encoding used (see the change in \S{}
\ref{subsubsection:changesRev4}). This octet should be interpreted as
an unsigned integer.

\item \label{pr:Fmt} \(Fmt\;::=\;'\{0x00\}\{0x02\}'\;|\;'\{0x06\}' \)

Format is a sequence that informs the parser whether the stream is
encoded in 8-bit format (like UTF-8) or a 16-bit format encoding (like
UTF-16). Note that this is not entirely dependent on character
encodings since there may be multiple character encodings in each
format. For example, the ISO-8859 family of encodings is 8-bit.

As an example, the xqML counterpart of the XML declaration: 
\begin{center} \verb|<?xml version="1.0" encoding="UTF-8"?>| \end{center}
looks like:
\begin{center} \verb*|    UTF-8| \end{center}
where \verb*| | 
is a visual representation of an xqML Symbol. The symbols in the above
example are 0x1E, 0x00, 0x02, and {\it Rev} -- in that sequence.
Each xqML revision number corresponds to a specific XML version
number. The special attribute ``standalone'' is not written and is
always assumed to be ``no''.

\item \label{pr:doctypedecl} \(doctypedecl\;::=\;('\{0x1E\}\{0x2C\}'\;DoctypeName)\;|\;xqA\;|\;DTDSect \)

An xqML document may declare its document type in one of three ways:
\begin{enumerate}
\item Declare a \textit{DoctypeName} (production
\ref{pr:DoctypeName}) that identifies an external xqA specification
\item Include an xqA specification (including prolog) inline
\item Include a DTD inline in a \textit{DTDSect} (production
\ref{pr:DTDSect})
\end{enumerate}

\item \label{pr:DoctypeName} \(DoctypeName\;::=\;Char* \)

\textit{DoctypeName} should be a valid URI from which an xqA
specification may be retrievable. However, the parser is not
responsible for checking the validity of a \textit{DoctypeName}.

\item \label{pr:DTDSect} \(DTDSect\;::=\;'\{0x1E\}\{0x2E\}'\;Char* \)

\textit{DTDSect} contains an internal DTD in the format specified in
XML 1.0 specification, including the \texttt{DOCTYPE} tag. An xqML
parser must be capable of generating an xqA specification out of the
DTD but is not always required to do so.

\item \label{pr:element} \(element\;::=\;NSDecl\!*\;RegId\!*\;STag\;(content\;ETag?)? \)

This corresponds to an XML Element. The element must have a start tag
\textit{STag}. The start tag also contains an indication of whether
the element is empty or not. If the element is not empty, it would
also contain \textit{content} and a closing tag.  The closing tag
\textit{ETag} is optional since several consecutive closing tags are
combined into one in xqML.

\item \label{pr:NSDecl} \(NSDecl\;::=\;'\{0x1E\}\{0x1C\}'\;Char*\;'\{0x1E\}'\;Char* \)

These are the xqML equivalents of xmlns declarations in XML. For
example, the declaration: \\
\verb|xmlns:xsl="http://www.w3.org/1999/XSL/Transform"| \\
would be encoded as:
\begin{center} \verb*|  xsl http://www.w3.org/1999/XSL/Transform| \end{center}
There may be a null string instead of ``xsl'' in the above example.

\item \label{pr:RegId} \(RegId\;::=\;'\{0x1E\}'\;('\{0x2A\}'\;|\;('\{0x28\}'\;NSSymbol))\;Char\!* \)

This production corresponds to an identifier registration in the
Dynamic Association mapping of the document (see
\S{}\ref{section:DynamicAssoc}). The string at the end of this
production is taken as the identifier to be registered. The
declaration may explicitly indicate association with a particular
namespace through the use of an \textit{NSSymbol}.

\item \label{pr:STag} \(STag\;::=\;'\{0x1E\}'\;(ELFlags\;NSSymbol?)?\;ELSymbol\;attribute\!* \)

This represents an element start tag. \textit{ELFlags} is an octet
that has three status flag bits. \textit{NSSymbol} is a symbol for XML
Namespace prefix. ELSymbol is the symbol for the element's
identifier. This may be followed by any number of attributes or XML
Namespace declarations (\textit{NSDecl}).

\item \label{pr:ELFlags} \(ELFlags\;::=\; 0x32 - 0x3E \)

This octet contains three status flags in its 2$^{nd}$, 3$^{rd}$ and
4$^{th}$ least significant bits  to signify the following:
\begin{enumerate}
\item {\it Empty Element}: The second least significant bit of the
  octet is set if the element is empty
\item {\it Namespace Prefix}: The third least significant bit is set
  if an {\it NSSymbol} follows
\item {\it Close Previous}: If the fourth least significant bit is
  set, it indicates that the last open element should be closed.
\end{enumerate}
The four most significant bits are \verb|0011|. Therefore this octet
can have values between 0x30 and 0x3E. However, if all the flag bits
are unset, the resultant value, 0x30, is never written. This value is
used to indicate one or more closing tags (See production
\ref{pr:ETag}).

\item \label{pr:attribute} \(attribute\;::=\;ATAttribute\;|\;APAttribute \)

Attributes may have unspecified values (\textit{ATAttribute}) or
values that have been assigned symbols in the xqA specification
(\textit{APAttribute}).

\item \label{pr:ATAttribute} \(ATAttribute\;::=\;(('\{0x14\}'\;NSSymbol)\;|\;'{0x16}')\;ATSymbol\;Char*\;(Reference\;Char*)*\;'\{0x16\}' \)

An attribute is started by the symbol 0x16, or by the symbol 0x14
followed by an {\it NSSymbol}. The symbol for the attribute
identifier, {\it ATSymbol}, comes next. The attribute is closed by the
symbol 0x16. Any character data or references before the closing
delimiter is taken to be the value of the attribute.

\item \label{pr:APAttribute} \(APAttribute\;::=\;(('\{0x18\}'\;NSSymbol)\;|\;'{0x1A}')\;APSymbol\;VASymbol \)

Attributes with predefined values begin with the symbol 0x18, or by
the symbol 0x1A followed by an {\it NSSymbol}. \textit{APSymbol} is
the symbol for the attribute identifier and \textit{VASymbol} is the
symbol for it's value. These attributes are completely represented by
symbols.

For example, the xqML counterpart of \\
\verb|<ufn:file path="/etc/issue.net" binary="no"/>|, where the
attribute ``binary'' has enumerated values ``yes'' and ``no'',  would
be:
\begin{center} \verb*|      /etc/issue.net    | \end{center}
Here we have six symbols, followed by the string ``/etc/issue.net''
followed by four more symbols. The symbols would be:
\begin{enumerate}
\item 0x1E
\item 0x36 (ELFlags, indicating an empty element and a namespace
  prefix to follow)
\item A document specific symbol for the namespace prefix ``ufn''
\item The symbol for element identifier ``file''
\item 0x16 -- to signify an attribute of type \textit{ATAttribute}
\item The symbol for attribute identifier ``path''
\end{enumerate}
The value of ``path'' follows as char data. The next four symbols
would be:
\begin{enumerate}
\item 0x16 -- to mark the end of attribute ``path''
\item 0x1A -- to signify an attribute of type \textit{APAttribute}
\item The symbol for attribute identifier ``binary''
\item The symbol for attribute value ``no''
\end{enumerate}

\item \label{pr:content} \(content\;::=\;Char*\;((element\;|\;Reference\;|\;PI)\;Char*)* \)

An element may contain character data and any number of child
elements, references or character data in any order. Restrictions
imposed by document type specifications (DTD, XML Schema etc.) may
apply while validating.

\item \label{pr:Reference} \(Reference\;::=\;EntityRef\;|\;CharRef \)

\item \label{pr:EntityRef} \(EntityRef\;::=\;'\{0x22\}'\;ENSymbol \)

This production matches an entity reference. ENSymbol is the symbol
for the entity identifier, \emph{not} its expansion.

\item \label{pr:CharRef} \(CharRef\;::=\;'\{0x1E\}\{0x26\}'\;VUint\; \)

This production matches a Character Reference. {\it VUint} is a
Variable-length Unsigned integer, whose value equals the code point of
the desired character.

\item \label{pr:ETag} \(ETag\;::=\;'\{0x1E\}\{0x30\}'\;ElementsToClose \)

The closing tag has an octet \textit{ElementsToClose} which should be
interpreted as the binary representation of an unsigned integer, whose
value signifies the number of elements to close in correct (stack)
order.

\item \label{pr:PI} \(PI\;::=\;'\{0x1E\}\{0x20\}'\;PITarget\;'\{0x1E\}'\;PIContent\;'\{0x1E\}' \)

This is a representation of an XML Processing
Instruction. \textit{PITarget} is the equivalent of targets in XML
PIs. \textit{PIContent} is the data that is passed on to the
application. For example, a hypothetical SSI include directive for a
web server may be written in XML as
\verb|<?ssi includefile("headers.shtml")?>|. The xqML equivalent of
this would be:
\begin{center} \verb*|  ssi includefile("headers.shtml") | \end{center}
where the symbols are 0x1E, 0x02, 0x1E and 0x1E in that order.

\item \label{pr:PITarget} \(PITarget\;::=\;Char* \) 

\item \label{pr:PIContent} \(PIContent\;::=\;Char* \)

\item \label{pr:Char} \(Char\;::=\;0x09\;|\;0x0A\;|\;0x0D\;|\;[0x20-0xD7FF]\;|\;[0xE00-0xFFFD]\;|\;[0x10000-0x10FFFF] \)

xqML characters are exactly same as XML characters. Additionaly, the
characters `$<$', `$>$', `\verb|'|', `\verb|"|' and `\&' need not be
escaped, unlike XML.

\end{enumerate}
