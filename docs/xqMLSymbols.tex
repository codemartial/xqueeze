%% Author: Tahir Hashmi
%% 
%% Copyright (C) 2002 2003, Xqueeze Developers
%% 
%% This file is part of Xqueeze Documentation
%% 
%% Permission is granted to copy, distribute and/or modify this
%% document under the terms of the GNU Free Documentation License,
%% Version 1.2 or any later version published by the Free Software
%% Foundation; with no Invariant Sections, no Front-Cover Texts, and no
%% Back-Cover Texts. A copy of the license is included in LaTeX source
%% format in the file entitled ``fdl.tex''.

\textit{xqML Symbols} are octect sequences that represent unsigned
integers written in Big-Endian (most significant byte first)
order. Additionally, the least significant bit of each octet of a
symbol, except the last one, should be one. For example, the integer
256 can be a valid xqML Symbol since when written in MSB order, its
bit pattern is [00000001 00000000]. Thus the least significant bit of
each octet in the symbol acts as a continuation flag. A 1 indicates
that the next octet is a part of the symbol, a 0 indicates the end.

It is trivially evident that all xqML Symbols must be even
numbers. Additionally, one bit of each octet is rendered unusable
since it acts as a continuation flag. 16 bit xqML Symbols can
represent 16384 different identifiers while 32 bit ones can represent
over 268 million different identifiers. A conforming implementation is
required to support atleast 16 bit long symbols.

xqML Symbols start from 0x02 (decimal 2) and symbols up to 0xFE
(decimal 254) are reserved for special purposes. Higher values are
available for generating associations using the Xqueeze Association
algorithm.

A special type of xqML Symbols -- {\it VUint} -- is defined to
represent variable length unsigned integral values that can
represent arbitrarily large values. The difference in interpretation
of the values of normal xqML Symbols and VUints is that the
continuation bits do not contribute to the value of the integer. For
example, while the xqML Symbol with decimal value 256 will be
represented as \verb|00000001 00000000|, a VUint will be represented
as \verb|00000101 00000000|. If we strip the continuation bits from
the latter, we get \verb|0000010 0000000| which represents decimal
256.

\subsubsection{Serialization}

xqML Symbols are serialized in Big-Endian (most significant byte
first) order and are represented in only as many octets (8-bit groups)
as required, irrespective of the encoding used for character
data. As an exception, some of the symbols should be serialized as
characters whose code points equal the value of the corresponding
symbol. These are the symbols with values 0x02, 0x06, 0x14, 0x16,
0x18, 0x1A, 0x1C and 0x1E.
 
