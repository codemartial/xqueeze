%% Author: Tahir Hashmi
%% 
%% Copyright (C) 2002 2003, Xqueeze Developers
%% 
%% This file is part of Xqueeze Documentation
%% 
%% Permission is granted to copy, distribute and/or modify this
%% document under the terms of the GNU Free Documentation License,
%% Version 1.2 or any later version published by the Free Software
%% Foundation; with no Invariant Sections, no Front-Cover Texts, and no
%% Back-Cover Texts. A copy of the license is included in LaTeX source
%% format in the file entitled ``fdl.tex''.

This is the allgorithm that is used to map the identifiers found in a
DTD/Schema to xqML Symbols. The steps of the algorithm are:
\begin{enumerate}
\item collect all Element identifiers
\item collect all Attribute identifiers
\item collect all Enumerated Attribute Value identifiers
\item collect all Entity References together
\item merge the above collections, discarding duplicates
\item sort the merged collection lexically on the values of unicode
  code-points
\item assign symbols starting from 256 in ascending order to the
  identifiers
\end{enumerate}

This simple algorithm assures that the assignments would remain the
same even if a particular specification (DTD/Schema) has slight
variations in the way it's written in the generator's and consumer's
copies, as long as both define the same vocabulary. Note that it is
not dependent on the structure of the document.

\subsubsection{Dynamic Associations} \label{section:DynamicAssoc}

Xqueeze allows for associating symbols to identifiers within a running
document through {\it Dynamic Associations}. This allows for
generation of xqML documents without the knowledge of whole or part of
the schema. Dynamic Associations cover elements, attributes and entity
references. Attribute values are not covered, and should be written as
string literals.

For assigning symbols to dynamically declared identifiers, the
processor must maintain a separate lookup table for each namespace
with which one or more dynamic identifier declarations are
associated. The namespace with which to associate a dynamically
declared identifier is determined by these rules:
\begin{enumerate}
\item Declarations appearing ahead of an element are associated with
  the namespace that the element is associated with
\item Declarations with explicit namespace prefixes are associated
  with the namespace denoted by the prefix, provided the prefix is
  valid and legal
\end{enumerate}

While registering identifiers dynamically, duplicate declarations
within the same namespace are discarded. This means that identifiers that
already exist in a given namespace would not be
re-assigned.\footnote{Therefore it is a good practice to declare
  dynamic identifiers in separate namespace(s) while mixing with
  various vocabularies.} Symbols are assigned to identifiers in the
order of their appearance in the document, starting from the first
unused symbol in the Association corresponding to the namespace in
context.

Portability of such associations is limited to the document that
contained the declarations and parts of the document using dynamically
assigned symbols can't be used elsewhere, without translation and
re-assigning of symbols. Nor can the document be safely modified
without preserving the declarations.
