%% Author: Tahir Hashmi
%% 
%% Copyright (C) 2002 - 2004, Xqueeze Developers
%% 
%% This file is part of Xqueeze Documentation
%% 
%% Permission is granted to copy, distribute and/or modify this
%% document under the terms of the GNU Free Documentation License,
%% Version 1.2 or any later version published by the Free Software
%% Foundation; with no Invariant Sections, no Front-Cover Texts, and no
%% Back-Cover Texts. A copy of the license is included in LaTeX source
%% format in the file entitled ``fdl.tex''.

Xqueeze associations are represented in a format that itself is quite
compact and uses xqML Symbols themselves. The specification begins
with an optional prolog whose format resembles that of an xqML
\textit{PI} (Processing Instruction): \\
\('\{0x1E\}\{0x20\}xqa\{0x1E\}'\;Char*\;'\{0x1E\}' \)

Here, \textit{Char*} may contain the identification string for the
document type. The prolog is followed by individual entries for
identifiers.

Individual entries are listed as `\{0x1E\}', followed by a symbol,
followed by a string that the symbol represents. The end of
specifications is denoted by the sequence `\{0x1E\}\{0x40\}'. This
structure enables inline specification of the symbols associations, if
required by a document.
