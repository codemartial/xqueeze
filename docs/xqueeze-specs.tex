%% Author: Tahir Hashmi
%% 
%% Copyright (C) 2002 2003, Xqueeze Developers
%% 
%% This file is part of Xqueeze Documentation
%% 
%% Permission is granted to copy, distribute and/or modify this
%% document under the terms of the GNU Free Documentation License,
%% Version 1.2 or any later version published by the Free Software
%% Foundation; with no Invariant Sections, no Front-Cover Texts, and no
%% Back-Cover Texts. A copy of the license is included in LaTeX source
%% format in the file entitled ``fdl.tex''.

\documentclass[a4paper]{article}
\usepackage{times}
\usepackage[pdfborder=111,colorlinks=true]{hyperref}
%% Author: Tahir Hashmi
%% 
%% Copyright (C) 2002 2003, Xqueeze Developers
%% 
%% This file is part of Xqueeze Documentation
%% 
%% Permission is granted to copy, distribute and/or modify this
%% document under the terms of the GNU Free Documentation License,
%% Version 1.2 or any later version published by the Free Software
%% Foundation; with no Invariant Sections, no Front-Cover Texts, and no
%% Back-Cover Texts. A copy of the license is included in LaTeX source
%% format in the file entitled ``fdl.tex''.

\newcommand{\RIver}{0.3}   %Reference Implementation Version No.
\newcommand{\libver}{3.0}  %libxqueeze Version No.
\newcommand{\xqAalgo}{0.2} %Xqueeze Association algo Version No.
\newcommand{\xqAfmt}{0.3}  %Xqueeze Association format Version No.
\newcommand{\xqMLver}{4}   %xqML Revision No.

\title{Xqueeze Specifications}
\author{Tahir Hashmi}
\begin{document}
\maketitle
\tableofcontents
\section{About this document}

\subsection{Copying}
\copyright{} 2003 Xqueeze Developers \\

Permission is granted to copy, distribute and/or modify this
document under the terms of the GNU Free Documentation License,
Version 1.2 or any later version published by the Free Software
Foundation; with no Invariant Sections, no Front-Cover Texts, and no
Back-Cover Texts. A copy of the license is included in section
\ref{section:gfdl} entitled ``GNU Free Documentation License''.

\subsection{Availability}
The latest version of this document can be downloaded from WWW by
pointing your browser to: \\
\href{http://xqueeze.sourceforge.net/xqueeze-specs.pdf}{http://xqueeze.sourceforge.net/xqueeze-specs.pdf}

Draft versions of this document can be downloaded from the CVS
Repository of Xqueeze project on Sourceforge.net by pointing your
browser to: \\
\href{http://cvs.sourceforge.net/cgi-bin/viewcvs.cgi/xqueeze/docs/}{http://cvs.sourceforge.net/cgi-bin/viewcvs.cgi/xqueeze/docs/}

\subsection{Terms of Use}
Usage of the \textbf{Specification} in this document for developing an
\textbf{Implementation} is subject to the following terms and conditions:
\begin{enumerate}
\item An \textbf{Implementation} advertizing itself to be based on the
  \textbf{Specification} should support the provisions of the
  \textbf{Specification} in whole or partially, provided it also
  fulfils condition 4.
\item An \textbf{Implementation} advertizing itself to be conforming
  to the Specification should put into effect the provisions of the
  \textbf{Specification} in whole.
\item The \textbf{Specifications} may be modified and re-distributed
  according to the provisions of GNU FDL (\S \ref{section:gfdl}).
\item An \textbf{Implementation} must not advertize itself to be based
  on or conforming to the \textbf{Specification} if it is based on a
  modification of the \textbf{Specification} unless such modifications
  are also distributed under the terms of GNU FDL (\S
  \ref{section:gfdl}) or compatible license.
\item Items 1 through 4 of these terms of use only apply to
  specifications and implementations that are distributed commercially
  or non-commercially to the public.
\end{enumerate}

\subsubsection*{Glossary}
\begin{description}
\item [Specification] The document or documents as designated by the
  Original Contributor that defines the form, interface and semantics
  to the technology covered by the contents of the Specification.
\item [Implementation] An implementation of the specification is a
  program or set of programs that puts into effect the form, interface
  and semantics defined by the Specification.
\item [Modification] Any deletion, addition or alteration to the form,
  interface or semantics defined by the Specification.
\item [Original Contributor] The initiator of the Specification.
\end{description}

\section{xqML language}

xqML is the binary markup language that is used by Xqueeze to achieve
compact document sizes as compared to XML documents. xqML is
structurally very similar to XML. The greatest contributors to xqML's
compact nature are the elimination of redundant information and
representation of XML identifiers (NMTOKENs) whose definitions are
available in the DTD/Schema with binary \textit{xqML Symbols}.

\subsection{xqML Symbols} \label{section:xqMLSymbols}
%% Author: Tahir Hashmi
%% 
%% Copyright (C) 2002 2003, Xqueeze Developers
%% 
%% This file is part of Xqueeze Documentation
%% 
%% Permission is granted to copy, distribute and/or modify this
%% document under the terms of the GNU Free Documentation License,
%% Version 1.2 or any later version published by the Free Software
%% Foundation; with no Invariant Sections, no Front-Cover Texts, and no
%% Back-Cover Texts. A copy of the license is included in LaTeX source
%% format in the file entitled ``fdl.tex''.

\textit{xqML Symbols} are octect sequences that represent unsigned
integers written in Big-Endian (most significant byte first)
order. Additionally, the least significant bit of each octet of a
symbol, except the last one, should be one. For example, the integer
256 can be a valid xqML Symbol since when written in MSB order, its
bit pattern is [00000001 00000000]. Thus the least significant bit of
each octet in the symbol acts as a continuation flag. A 1 indicates
that the next octet is a part of the symbol, a 0 indicates the end.

It is trivially evident that all xqML Symbols must be even
numbers. Additionally, one bit of each octet is rendered unusable
since it acts as a continuation flag. 16 bit xqML Symbols can
represent 16384 different identifiers while 32 bit ones can represent
over 268 million different identifiers. A conforming implementation is
required to support atleast 16 bit long symbols.

xqML Symbols start from 0x02 (decimal 2) and symbols up to 0xFE
(decimal 254) are reserved for special purposes. Higher values are
available for generating associations using the Xqueeze Association
algorithm.

A special type of xqML Symbols -- {\it VUint} -- is defined to
represent variable length unsigned integral values that can
represent arbitrarily large values. The difference in interpretation
of the values of normal xqML Symbols and VUints is that the
continuation bits do not contribute to the value of the integer. For
example, while the xqML Symbol with decimal value 256 will be
represented as \verb|00000001 00000000|, a VUint will be represented
as \verb|00000101 00000000|. If we strip the continuation bits from
the latter, we get \verb|0000010 0000000| which represents decimal
256.

\subsubsection{Serialization}

xqML Symbols are serialized in Big-Endian (most significant byte
first) order and are represented in only as many octets (8-bit groups)
as required, irrespective of the encoding used for character
data. As an exception, some of the symbols should be serialized as
characters whose code points equal the value of the corresponding
symbol. These are the symbols with values 0x02, 0x06, 0x14, 0x16,
0x18, 0x1A, 0x1C and 0x1E.
 


%% Author: Tahir Hashmi
%% 
%% Copyright (C) 2002 2003, Xqueeze Developers
%% 
%% This file is part of Xqueeze Documentation
%% 
%% Permission is granted to copy, distribute and/or modify this
%% document under the terms of the GNU Free Documentation License,
%% Version 1.2 or any later version published by the Free Software
%% Foundation; with no Invariant Sections, no Front-Cover Texts, and no
%% Back-Cover Texts. A copy of the license is included in LaTeX source
%% format in the file entitled ``fdl.tex''.

\subsubsection{Reserved Symbols}

xqML Symbols with values between 0x02 and 0xFE (both inclusive) are
reserved for grammar specific
purposes. Table~\ref{table:reservedSymbols} lists the used xqML
Symbols in xqML revision \xqMLver{} and their purpose. Entries in
\textit{italics} are productions from the xqML grammar listed in \S
\ref{section:xqml-gram} and the respective production numbers are
provided in brackets.

\begin{table}[!hbp]
\begin{tabular}{|c|c|l|}
\hline
\multicolumn{2}{|c|}{\textbf{Symbol Value}} &
\multicolumn{1}{|c|}{\textbf{Purpose}} \\
\cline{1-2} \textbf{Hex} & \textbf{Dec} & \\ \hline \hline

0x02 & 002 &   \textit{Fmt} (\ref{pr:Fmt})\\ \hline
0x04 & 004 &   \\ \hline
0x06 & 006 &   \textit{Fmt} (\ref{pr:Fmt})\\ \hline
%% 0x08 & 008 &   \\ \hline
%% 0x0A & 010 &   \\ \hline
%% 0x0C & 012 &   \\ \hline
%% 0x0E & 014 &   \\ \hline
%% 0x10 & 016 &   \\ \hline
%% 0x12 & 018 &   \\ \hline
\hline % To graphically signify a break in numbering
0x14 & 020 &   \textit{ATAttribute} (\ref{pr:ATAttribute})\\ \hline
0x16 & 022 &   \textit{ATAttribute} (\ref{pr:ATAttribute})\\ \hline
0x18 & 024 &   \textit{APAttribute} (\ref{pr:APAttribute})\\ \hline
0x1A & 026 &   \textit{APAttribute} (\ref{pr:APAttribute})\\ \hline
0x1C & 028 &   \textit{NSDecl} (\ref{pr:NSDecl})\\ \hline
0x1E & 030 &   Markup flag\\ \hline
0x20 & 032 &   \textit{PI} (\ref{pr:PI})\\ \hline
0x22 & 034 &   \textit{EntityRef} (\ref{pr:EntityRef})\\ \hline
%% 0x24 & 036 &   \textit{EntityRef} (\ref{pr:EntityRef})\\ \hline
0x24 & 036 &   \\ \hline
0x26 & 038 &   \textit{CharRef} (\ref{pr:CharRef})\\ \hline
0x28 & 040 &   \textit{RegId} (\ref{pr:RegId})\\ \hline
0x2A & 042 &   \textit{RegId} (\ref{pr:RegId})\\ \hline
0x2C & 044 &   \textit{doctypedecl} (\ref{pr:doctypedecl})\\ \hline
0x2E & 046 &   \textit{DTDSect} (\ref{pr:DTDSect})\\ \hline
0x30 & 048 &   \textit{ETag} (\ref{pr:ETag})\\ \hline
0x32 & 050 &   \textit{ELFlags} (\ref{pr:ELFlags})\\ \hline
0x34 & 052 &   \textit{ELFlags} (\ref{pr:ELFlags})\\ \hline
0x36 & 054 &   \textit{ELFlags} (\ref{pr:ELFlags})\\ \hline
0x38 & 056 &   \textit{ELFlags} (\ref{pr:ELFlags})\\ \hline
0x3A & 058 &   \textit{ELFlags} (\ref{pr:ELFlags})\\ \hline
0x3C & 060 &   \textit{ELFlags} (\ref{pr:ELFlags})\\ \hline
0x3E & 062 &   \textit{ELFlags} (\ref{pr:ELFlags})\\ \hline
0x40 & 064 &   xqA end marker\\ \hline
%% \hline
%% \multicolumn{3}{|l|}{The remaining symbols are unutilized so far} \\
%% \hline
\end{tabular}
\caption{Table of Reserved Symbols}
\label{table:reservedSymbols}
\end{table}


\subsection{xqML Grammar (Revision \xqMLver{})} \label{section:xqml-gram}
%% Author: Tahir Hashmi
%% 
%% Copyright (C) 2002 2003, Xqueeze Developers
%% 
%% This file is part of Xqueeze Documentation
%% 
%% Permission is granted to copy, distribute and/or modify this
%% document under the terms of the GNU Free Documentation License,
%% Version 1.2 or any later version published by the Free Software
%% Foundation; with no Invariant Sections, no Front-Cover Texts, and no
%% Back-Cover Texts. A copy of the license is included in LaTeX source
%% format in the file entitled ``fdl.tex''.

\subsubsection{Terminals}
\begin{itemize}
\item \textit{Figures enclosed within braces ($\{\}$)} are hex codes for the value
  of an xqML Symbol that should occur within.
\item \textit{Rev} is an octet to be interpreted as an unsigned integer.
\item \textit{xqA} is the inline Xqueeze Association with prolog
  (see \S{} \ref{section:xqa-fmt}).
\item \textit{ELSymbol}, \textit{ATSymbol}, \textit{APSymbol},
  \textit{VASymbol} and \textit{ENSymbol} are all xqML symbols derived
  from an Xqueeze Association to represent the vocabulary of an XML
  document type.
\item \textit{NSSymbol} is an xqML Symbol of the type ``namespace prefix''
  (has a document-specific value).
\item \textit{ElementsToClose} is an octet to be interpreted as an unsigned
  integer.
\item \textit{VUint} is a special type of xqML Symbols that represents
  Variable-length Unsigned integers (see \S{} \ref{section:xqMLSymbols}).
\end{itemize}

\subsubsection{Productions}
\begin{enumerate}
\item \label{pr:document} \(document\;::=\;prolog\;element\;PI* \)

Every xqML document must match the above production. Thus,
\textit{document} is the starting symbol.

\item \label{pr:prolog} \(prolog\;::=\;[^\wedge\{0x1E\}]\!*\;xqMLDecl\;PI\!*\;(doctypedecl\;PI*)? \)

The \textit{prolog} of an xqML document can contain anything upto the
first occurrence of xqML Symbol \{0x1E\}.

\item \label{pr:xqMLDecl} \(xqMLDecl\;::=\;'\{0x1E\}'\;Fmt\;Rev\;Char* \)

Every xqML document must declare what it is (xqML), its binary format
and the version of its encoding. {\it Rev} is an octet that represents
the revision number of the xqML encoding used (see the change in \S{}
\ref{subsubsection:changesRev4}). This octet should be interpreted as
an unsigned integer.

\item \label{pr:Fmt} \(Fmt\;::=\;'\{0x00\}\{0x02\}'\;|\;'\{0x06\}' \)

Format is a sequence that informs the parser whether the stream is
encoded in 8-bit format (like UTF-8) or a 16-bit format encoding (like
UTF-16). Note that this is not entirely dependent on character
encodings since there may be multiple character encodings in each
format. For example, the ISO-8859 family of encodings is 8-bit.

As an example, the xqML counterpart of the XML declaration: 
\begin{center} \verb|<?xml version="1.0" encoding="UTF-8"?>| \end{center}
looks like:
\begin{center} \verb*|    UTF-8| \end{center}
where \verb*| | 
is a visual representation of an xqML Symbol. The symbols in the above
example are 0x1E, 0x00, 0x02, and {\it Rev} -- in that sequence.
Each xqML revision number corresponds to a specific XML version
number. The special attribute ``standalone'' is not written and is
always assumed to be ``no''.

\item \label{pr:doctypedecl} \(doctypedecl\;::=\;('\{0x1E\}\{0x2C\}'\;DoctypeName)\;|\;xqA\;|\;DTDSect \)

An xqML document may declare its document type in one of three ways:
\begin{enumerate}
\item Declare a \textit{DoctypeName} (production
\ref{pr:DoctypeName}) that identifies an external xqA specification
\item Include an xqA specification (including prolog) inline
\item Include a DTD inline in a \textit{DTDSect} (production
\ref{pr:DTDSect})
\end{enumerate}

\item \label{pr:DoctypeName} \(DoctypeName\;::=\;Char* \)

\textit{DoctypeName} should be a valid URI from which an xqA
specification may be retrievable. However, the parser is not
responsible for checking the validity of a \textit{DoctypeName}.

\item \label{pr:DTDSect} \(DTDSect\;::=\;'\{0x1E\}\{0x2E\}'\;Char* \)

\textit{DTDSect} contains an internal DTD in the format specified in
XML 1.0 specification, including the \texttt{DOCTYPE} tag. An xqML
parser must be capable of generating an xqA specification out of the
DTD but is not always required to do so.

\item \label{pr:element} \(element\;::=\;NSDecl\!*\;RegId\!*\;STag\;(content\;ETag?)? \)

This corresponds to an XML Element. The element must have a start tag
\textit{STag}. The start tag also contains an indication of whether
the element is empty or not. If the element is not empty, it would
also contain \textit{content} and a closing tag.  The closing tag
\textit{ETag} is optional since several consecutive closing tags are
combined into one in xqML.

\item \label{pr:NSDecl} \(NSDecl\;::=\;'\{0x1E\}\{0x1C\}'\;Char*\;'\{0x1E\}'\;Char* \)

These are the xqML equivalents of xmlns declarations in XML. For
example, the declaration: \\
\verb|xmlns:xsl="http://www.w3.org/1999/XSL/Transform"| \\
would be encoded as:
\begin{center} \verb*|  xsl http://www.w3.org/1999/XSL/Transform| \end{center}
There may be a null string instead of ``xsl'' in the above example.

\item \label{pr:RegId} \(RegId\;::=\;'\{0x1E\}'\;('\{0x2A\}'\;|\;('\{0x28\}'\;NSSymbol))\;Char\!* \)

This production corresponds to an identifier registration in the
Dynamic Association mapping of the document (see
\S{}\ref{section:DynamicAssoc}). The string at the end of this
production is taken as the identifier to be registered. The
declaration may explicitly indicate association with a particular
namespace through the use of an \textit{NSSymbol}.

\item \label{pr:STag} \(STag\;::=\;'\{0x1E\}'\;(ELFlags\;NSSymbol?)?\;ELSymbol\;attribute\!* \)

This represents an element start tag. \textit{ELFlags} is an octet
that has three status flag bits. \textit{NSSymbol} is a symbol for XML
Namespace prefix. ELSymbol is the symbol for the element's
identifier. This may be followed by any number of attributes or XML
Namespace declarations (\textit{NSDecl}).

\item \label{pr:ELFlags} \(ELFlags\;::=\; 0x32 - 0x3E \)

This octet contains three status flags in its 2$^{nd}$, 3$^{rd}$ and
4$^{th}$ least significant bits  to signify the following:
\begin{enumerate}
\item {\it Empty Element}: The second least significant bit of the
  octet is set if the element is empty
\item {\it Namespace Prefix}: The third least significant bit is set
  if an {\it NSSymbol} follows
\item {\it Close Previous}: If the fourth least significant bit is
  set, it indicates that the last open element should be closed.
\end{enumerate}
The four most significant bits are \verb|0011|. Therefore this octet
can have values between 0x30 and 0x3E. However, if all the flag bits
are unset, the resultant value, 0x30, is never written. This value is
used to indicate one or more closing tags (See production
\ref{pr:ETag}).

\item \label{pr:attribute} \(attribute\;::=\;ATAttribute\;|\;APAttribute \)

Attributes may have unspecified values (\textit{ATAttribute}) or
values that have been assigned symbols in the xqA specification
(\textit{APAttribute}).

\item \label{pr:ATAttribute} \(ATAttribute\;::=\;(('\{0x14\}'\;NSSymbol)\;|\;'{0x16}')\;ATSymbol\;Char*\;(Reference\;Char*)*\;'\{0x16\}' \)

An attribute is started by the symbol 0x16, or by the symbol 0x14
followed by an {\it NSSymbol}. The symbol for the attribute
identifier, {\it ATSymbol}, comes next. The attribute is closed by the
symbol 0x16. Any character data or references before the closing
delimiter is taken to be the value of the attribute.

\item \label{pr:APAttribute} \(APAttribute\;::=\;(('\{0x18\}'\;NSSymbol)\;|\;'{0x1A}')\;APSymbol\;VASymbol \)

Attributes with predefined values begin with the symbol 0x18, or by
the symbol 0x1A followed by an {\it NSSymbol}. \textit{APSymbol} is
the symbol for the attribute identifier and \textit{VASymbol} is the
symbol for it's value. These attributes are completely represented by
symbols.

For example, the xqML counterpart of \\
\verb|<ufn:file path="/etc/issue.net" binary="no"/>|, where the
attribute ``binary'' has enumerated values ``yes'' and ``no'',  would
be:
\begin{center} \verb*|      /etc/issue.net    | \end{center}
Here we have six symbols, followed by the string ``/etc/issue.net''
followed by four more symbols. The symbols would be:
\begin{enumerate}
\item 0x1E
\item 0x36 (ELFlags, indicating an empty element and a namespace
  prefix to follow)
\item A document specific symbol for the namespace prefix ``ufn''
\item The symbol for element identifier ``file''
\item 0x16 -- to signify an attribute of type \textit{ATAttribute}
\item The symbol for attribute identifier ``path''
\end{enumerate}
The value of ``path'' follows as char data. The next four symbols
would be:
\begin{enumerate}
\item 0x16 -- to mark the end of attribute ``path''
\item 0x1A -- to signify an attribute of type \textit{APAttribute}
\item The symbol for attribute identifier ``binary''
\item The symbol for attribute value ``no''
\end{enumerate}

\item \label{pr:content} \(content\;::=\;Char*\;((element\;|\;Reference\;|\;PI)\;Char*)* \)

An element may contain character data and any number of child
elements, references or character data in any order. Restrictions
imposed by document type specifications (DTD, XML Schema etc.) may
apply while validating.

\item \label{pr:Reference} \(Reference\;::=\;EntityRef\;|\;CharRef \)

\item \label{pr:EntityRef} \(EntityRef\;::=\;'\{0x22\}'\;ENSymbol \)

This production matches an entity reference. ENSymbol is the symbol
for the entity identifier, \emph{not} its expansion.

\item \label{pr:CharRef} \(CharRef\;::=\;'\{0x1E\}\{0x26\}'\;VUint\; \)

This production matches a Character Reference. {\it VUint} is a
Variable-length Unsigned integer, whose value equals the code point of
the desired character.

\item \label{pr:ETag} \(ETag\;::=\;'\{0x1E\}\{0x30\}'\;ElementsToClose \)

The closing tag has an octet \textit{ElementsToClose} which should be
interpreted as the binary representation of an unsigned integer, whose
value signifies the number of elements to close in correct (stack)
order.

\item \label{pr:PI} \(PI\;::=\;'\{0x1E\}\{0x20\}'\;PITarget\;'\{0x1E\}'\;PIContent\;'\{0x1E\}' \)

This is a representation of an XML Processing
Instruction. \textit{PITarget} is the equivalent of targets in XML
PIs. \textit{PIContent} is the data that is passed on to the
application. For example, a hypothetical SSI include directive for a
web server may be written in XML as
\verb|<?ssi includefile("headers.shtml")?>|. The xqML equivalent of
this would be:
\begin{center} \verb*|  ssi includefile("headers.shtml") | \end{center}
where the symbols are 0x1E, 0x02, 0x1E and 0x1E in that order.

\item \label{pr:PITarget} \(PITarget\;::=\;Char* \) 

\item \label{pr:PIContent} \(PIContent\;::=\;Char* \)

\item \label{pr:Char} \(Char\;::=\;0x09\;|\;0x0A\;|\;0x0D\;|\;[0x20-0xD7FF]\;|\;[0xE00-0xFFFD]\;|\;[0x10000-0x10FFFF] \)

xqML characters are exactly same as XML characters. Additionaly, the
characters `$<$', `$>$', `\verb|'|', `\verb|"|' and `\&' need not be
escaped, unlike XML.

\end{enumerate}


\section{Xqueeze Association}

Xqueeze uses an association between symbols and their corresponding
XML identifiers and types as defined in a specification
(DTD/Schema). This enables representation of known identifiers in the
markup with symbols. Associating the type of an identifier along with
it's name also makes it easy to various structural units of the
document without having to use too many special characters and
character-combinations.

\subsection{Xqueeze Association Algorithm (Version \xqAalgo{})} \label{section:xqa-algo}
%% Author: Tahir Hashmi
%% 
%% Copyright (C) 2002 2003, Xqueeze Developers
%% 
%% This file is part of Xqueeze Documentation
%% 
%% Permission is granted to copy, distribute and/or modify this
%% document under the terms of the GNU Free Documentation License,
%% Version 1.2 or any later version published by the Free Software
%% Foundation; with no Invariant Sections, no Front-Cover Texts, and no
%% Back-Cover Texts. A copy of the license is included in LaTeX source
%% format in the file entitled ``fdl.tex''.

This is the allgorithm that is used to map the identifiers found in a
DTD/Schema to xqML Symbols. The steps of the algorithm are:
\begin{enumerate}
\item collect all Element identifiers
\item collect all Attribute identifiers
\item collect all Enumerated Attribute Value identifiers
\item collect all Entity References together
\item merge the above collections, discarding duplicates
\item sort the merged collection lexically on the values of unicode
  code-points
\item assign symbols starting from 256 in ascending order to the
  identifiers
\end{enumerate}

This simple algorithm assures that the assignments would remain the
same even if a particular specification (DTD/Schema) has slight
variations in the way it's written in the generator's and consumer's
copies, as long as both define the same vocabulary. Note that it is
not dependent on the structure of the document.

\subsubsection{Dynamic Associations} \label{section:DynamicAssoc}

Xqueeze allows for associating symbols to identifiers within a running
document through {\it Dynamic Associations}. This allows for
generation of xqML documents without the knowledge of whole or part of
the schema. Dynamic Associations cover elements, attributes and entity
references. Attribute values are not covered, and should be written as
string literals.

For assigning symbols to dynamically declared identifiers, the
processor must maintain a separate lookup table for each namespace
with which one or more dynamic identifier declarations are
associated. The namespace with which to associate a dynamically
declared identifier is determined by these rules:
\begin{enumerate}
\item Declarations appearing ahead of an element are associated with
  the namespace that the element is associated with
\item Declarations with explicit namespace prefixes are associated
  with the namespace denoted by the prefix, provided the prefix is
  valid and legal
\end{enumerate}

While registering identifiers dynamically, duplicate declarations
within the same namespace are discarded. This means that identifiers that
already exist in a given namespace would not be
re-assigned.\footnote{Therefore it is a good practice to declare
  dynamic identifiers in separate namespace(s) while mixing with
  various vocabularies.} Symbols are assigned to identifiers in the
order of their appearance in the document, starting from the first
unused symbol in the Association corresponding to the namespace in
context.

Portability of such associations is limited to the document that
contained the declarations and parts of the document using dynamically
assigned symbols can't be used elsewhere, without translation and
re-assigning of symbols. Nor can the document be safely modified
without preserving the declarations.


\subsection{Xqueeze Association Format (Version \xqAfmt{})} \label{section:xqa-fmt}
%% Author: Tahir Hashmi
%% 
%% Copyright (C) 2002 - 2004, Xqueeze Developers
%% 
%% This file is part of Xqueeze Documentation
%% 
%% Permission is granted to copy, distribute and/or modify this
%% document under the terms of the GNU Free Documentation License,
%% Version 1.2 or any later version published by the Free Software
%% Foundation; with no Invariant Sections, no Front-Cover Texts, and no
%% Back-Cover Texts. A copy of the license is included in LaTeX source
%% format in the file entitled ``fdl.tex''.

Xqueeze associations are represented in a format that itself is quite
compact and uses xqML Symbols themselves. The specification begins
with an optional prolog whose format resembles that of an xqML
\textit{PI} (Processing Instruction): \\
\('\{0x1E\}\{0x20\}xqa\{0x1E\}'\;Char*\;'\{0x1E\}' \)

Here, \textit{Char*} may contain the identification string for the
document type. The prolog is followed by individual entries for
identifiers.

Individual entries are listed as `\{0x1E\}', followed by a symbol,
followed by a string that the symbol represents. The end of
specifications is denoted by the sequence `\{0x1E\}\{0x40\}'. This
structure enables inline specification of the symbols associations, if
required by a document.


\section{Changes}

\subsection{xqML}
\subsubsection{Revision 4} \label{subsubsection:changesRev4}
\begin{itemize}
\item xqML will now have ``Revisions'' instead of version numbers. The
  current format can report a maximum of 255 revisions. However, this
  does not imply that there will not be more than 255 revisions of
  xqML
\item Comments have now been dropped
\item CDATA Sections have now been dropped
\item The format now allows for generation of documents without prior
  knowledge of schema through {\it Dynamic Associations} (\S{}
  \ref{pr:RegId})
\item The {\it xqMLDecl} represents xqML revision information in
  binary now
\item A new terminal {\it Rev} has been added
\item {\it xqMLDecl} is now mandatory for all xqML documents
\item All xqML documents have the value of special attribute
  ``standalone'' as ``no''
\item {\it ELFlags} production added to combine three flags related to
  elements into one octet
\item {\it ATAttribute} ends with `\{0x16\}' instead of
  `\{0x1E\}\{0x16\}'
\item A new terminal and xqML Symbol type, {\it VUint} has been added
  (see \S{} \ref{section:xqMLSymbols})
\item {\it CharRef} now uses VUint to encode the character's code
  point value
\item The production {\it EE\_STag} has been dropped
\item {\it ETag} now uses `\{0x30\}' instead of `\{0x3E\}'.
\end{itemize}
\subsubsection{Version 0.3}
\begin{itemize}
\item Anything is permissible upto the occurrence of \textit{xqMLDecl}
  in a document
\item A new production, \textit{PI}, has been added for Processing
  Instructions
\item \textit{doctypedecl} now starts with `\{0x1E\}\{0x2C\}' instead
  of `\{0x1E\}\{0x12\}'
\item \textit{xqA} should necessarily include a prolog now
\item \textit{doctypedecl} may now have an inline DTD with a new
  production \textit{DTDSect}.
\item \textit{element} production was erroneous till the last
  version
\item A new prodcution \textit{NSPrefix} has been added for XML
  Namespace prefixes
\item The productions \textit{EE\_STag}, \textit{STag},
  \textit{ATAttribute}, \textit{APAttribute} and \textit{EntityRef}
  can now have namespace prefixes
\item \textit{EE\_STag} starts with `\{0x1E\}\{0x2A\}' instead of
  `\{0x1E\}'
\item \textit{ATAttribute} starts with `\{0x16\}' instead of
  `\{0x1E\}'
\item \textit{APAttribute} starts with `\{0x18\}' instead of
  `\{0x1E\}'
\item \textit{EntityRef} starts with `\{0x1E\}\{0x24\}' instead of
  `\{0x1E\}'
\item \textit{CDSect} starts with `\{0x1E\}\{0x28\}' instead of
  \textit{CDDelim} and ends with `\{0x1E\}' instead of
  \textit{CDDelim} (\{0x1E\}\{0x14\}).
\item \textit{Char} now matches the \textit{Char} production in XML
  1.0 grammar specification.
\end{itemize}
\subsubsection{Version 0.2}
\begin{itemize}
\item xqML Symbol `\{0x1E\}' replaces `$<$' for the latter's role in
  xqML markup
\item \textit{Attribute} is split into \textit{ATAttribute} and
  \textit{APAttribute}, together referred as \textit{attribute}.
\item \textit{ATAttribute} can contain \textit{Reference}.
\item \textit{ATAttribute} is terminated by `\{0x1E\}\{0x16\}' instead of
  `$<$'
\item \textit{CharRef} starts with `\{0x1E\}\{0x26\}' instead of
  `\&\{0x26\}'
\item \textit{CharRef} ends with `\{0x1E\}'. Earlier there was no
  end-marker
\item \textit{Comment} ends with `\{0x1E\}' instead of \textit{ETag?}
\item \textit{Comment} is deprecated
\item \textit{Char} is a terminal that matches any printable character
\item \textit{Num} does not contain `.'
\end{itemize}
\subsubsection{Version 0.1}
First Release

\subsection{Xqueeze Association algorithm}
\subsubsection{Version 0.2}
\begin{itemize}
\item Removed distinction of identifiers based on type
\item Added support for Dynamic Associations
\end{itemize}
\subsubsection{Version 0.1}
First Release

\subsection{Xqueeze Association format}
\subsubsection{Version 0.3}
\begin{itemize}
\item Removed section markers
\item xqA specification now ends with the sequence `\{0x1E\}\{0x40\}'
  instead of `\{0x1E\}\{0x3C\}'.
\end{itemize}
\subsubsection{Version 0.2}
\begin{itemize}
\item xqML Symbol `\{0x1E\}' replaces `$<$' for the latter's role in
  xqA format.
\item The prolog format has been changed to resemble an xqML PI.
\item Reserved symbols used in the previous version have been shifted
  44 decimal values up. For example, the symbol for Element section is
  now `\{0x30\}' (48) instead of `\{0x04\}' (04).
\end{itemize}
\subsubsection{Version 0.1}
First Release

\section{GNU Free Documentation License} \label{section:gfdl}
\begin{scriptsize}
\input{fdl.tex}
\end{scriptsize}

\end{document}
