%% Author: Tahir Hashmi
%% 
%% Copyright (C) 2002 2003, Xqueeze Developers
%% 
%% This file is part of Xqueeze Documentation
%% 
%% Permission is granted to copy, distribute and/or modify this
%% document under the terms of the GNU Free Documentation License,
%% Version 1.2 or any later version published by the Free Software
%% Foundation; with no Invariant Sections, no Front-Cover Texts, and no
%% Back-Cover Texts. A copy of the license is included in LaTeX source
%% format in the file entitled ``fdl.tex''.

\documentclass[a4paper]{article}
\usepackage{times}
\usepackage[pdfborder=111,colorlinks=true]{hyperref}
%% Author: Tahir Hashmi
%% 
%% Copyright (C) 2002 2003, Xqueeze Developers
%% 
%% This file is part of Xqueeze Documentation
%% 
%% Permission is granted to copy, distribute and/or modify this
%% document under the terms of the GNU Free Documentation License,
%% Version 1.2 or any later version published by the Free Software
%% Foundation; with no Invariant Sections, no Front-Cover Texts, and no
%% Back-Cover Texts. A copy of the license is included in LaTeX source
%% format in the file entitled ``fdl.tex''.

\newcommand{\RIver}{0.3}   %Reference Implementation Version No.
\newcommand{\libver}{3.0}  %libxqueeze Version No.
\newcommand{\xqAalgo}{0.2} %Xqueeze Association algo Version No.
\newcommand{\xqAfmt}{0.3}  %Xqueeze Association format Version No.
\newcommand{\xqMLver}{4}   %xqML Revision No.

\title{Xqueeze Specifications}
\author{Tahir Hashmi}
\begin{document}
\maketitle
\tableofcontents
\section{About this document}

\subsection{Copying}
\copyright{} 2003 Xqueeze Developers \\

Permission is granted to copy, distribute and/or modify this
document under the terms of the GNU Free Documentation License,
Version 1.2 or any later version published by the Free Software
Foundation; with no Invariant Sections, no Front-Cover Texts, and no
Back-Cover Texts. A copy of the license is included in section
\ref{section:gfdl} entitled ``GNU Free Documentation License''.

\subsection{Availability}
The latest version of this document can be downloaded from WWW by
pointing your browser to: \\
\href{http://xqueeze.sourceforge.net/xqueeze-specs.pdf}{http://xqueeze.sourceforge.net/xqueeze-specs.pdf}

Draft versions of this document can be downloaded from the CVS
Repository of Xqueeze project on Sourceforge.net by pointing your
browser to: \\
\href{http://cvs.sourceforge.net/cgi-bin/viewcvs.cgi/xqueeze/docs/}{http://cvs.sourceforge.net/cgi-bin/viewcvs.cgi/xqueeze/docs/}

\subsection{Terms of Use}
Usage of the \textbf{Specification} in this document for developing an
\textbf{Implementation} is subject to the following terms and conditions:
\begin{enumerate}
\item An \textbf{Implementation} advertizing itself to be based on the
  \textbf{Specification} should support the provisions of the
  \textbf{Specification} in whole or partially, provided it also
  fulfils condition 4.
\item An \textbf{Implementation} advertizing itself to be conforming
  to the Specification should put into effect the provisions of the
  \textbf{Specification} in whole.
\item The \textbf{Specifications} may be modified and re-distributed
  according to the provisions of GNU FDL (\S \ref{section:gfdl}).
\item An \textbf{Implementation} must not advertize itself to be based
  on or conforming to the \textbf{Specification} if it is based on a
  modification of the \textbf{Specification} unless such modifications
  are also distributed under the terms of GNU FDL (\S
  \ref{section:gfdl}) or compatible license.
\item Items 1 through 4 of these terms of use only apply to
  specifications and implementations that are distributed commercially
  or non-commercially to the public.
\end{enumerate}

\subsubsection*{Glossary}
\begin{description}
\item [Specification] The document or documents as designated by the
  Original Contributor that defines the form, interface and semantics
  to the technology covered by the contents of the Specification.
\item [Implementation] An implementation of the specification is a
  program or set of programs that puts into effect the form, interface
  and semantics defined by the Specification.
\item [Modification] Any deletion, addition or alteration to the form,
  interface or semantics defined by the Specification.
\item [Original Contributor] The initiator of the Specification.
\end{description}

\section{xqML language}

xqML is the binary markup language that is used by Xqueeze to achieve
compact document sizes as compared to XML documents. xqML is
structurally very similar to XML. The greatest contributors to xqML's
compact nature are the elimination of redundant information and
representation of XML identifiers (NMTOKENs) whose definitions are
available in the DTD/Schema with binary \textit{xqML Symbols}.

\subsection{xqML Symbols} \label{section:xqMLSymbols}
%% Author: Tahir Hashmi
%% 
%% Copyright (C) 2002 2003, Xqueeze Developers
%% 
%% This file is part of Xqueeze Documentation
%% 
%% Permission is granted to copy, distribute and/or modify this
%% document under the terms of the GNU Free Documentation License,
%% Version 1.2 or any later version published by the Free Software
%% Foundation; with no Invariant Sections, no Front-Cover Texts, and no
%% Back-Cover Texts. A copy of the license is included in LaTeX source
%% format in the file entitled ``fdl.tex''.

\textit{xqML Symbols} are octect sequences that represent unsigned
integers written in Big-Endian (most significant byte first)
order. Additionally, the least significant bit of each octet of a
symbol, except the last one, should be one. For example, the integer
256 can be a valid xqML Symbol since when written in MSB order, its
bit pattern is [00000001 00000000]. Thus the least significant bit of
each octet in the symbol acts as a continuation flag. A 1 indicates
that the next octet is a part of the symbol, a 0 indicates the end.

It is trivially evident that all xqML Symbols must be even
numbers. Additionally, one bit of each octet is rendered unusable
since it acts as a continuation flag. 16 bit xqML Symbols can
represent 16384 different identifiers while 32 bit ones can represent
over 268 million different identifiers. A conforming implementation is
required to support atleast 16 bit long symbols.

xqML Symbols start from 0x02 (decimal 2) and symbols up to 0xFE
(decimal 254) are reserved for special purposes. Higher values are
available for generating associations using the Xqueeze Association
algorithm.

A special type of xqML Symbols -- {\it VUint} -- is defined to
represent variable length unsigned integral values that can
represent arbitrarily large values. The difference in interpretation
of the values of normal xqML Symbols and VUints is that the
continuation bits do not contribute to the value of the integer. For
example, while the xqML Symbol with decimal value 256 will be
represented as \verb|00000001 00000000|, a VUint will be represented
as \verb|00000101 00000000|. If we strip the continuation bits from
the latter, we get \verb|0000010 0000000| which represents decimal
256.

\subsubsection{Serialization}

xqML Symbols are serialized in Big-Endian (most significant byte
first) order and are represented in only as many octets (8-bit groups)
as required, irrespective of the encoding used for character
data. As an exception, some of the symbols should be serialized as
characters whose code points equal the value of the corresponding
symbol. These are the symbols with values 0x02, 0x06, 0x14, 0x16,
0x18, 0x1A, 0x1C and 0x1E.
 


%% Author: Tahir Hashmi
%% 
%% Copyright (C) 2002 2003, Xqueeze Developers
%% 
%% This file is part of Xqueeze Documentation
%% 
%% Permission is granted to copy, distribute and/or modify this
%% document under the terms of the GNU Free Documentation License,
%% Version 1.2 or any later version published by the Free Software
%% Foundation; with no Invariant Sections, no Front-Cover Texts, and no
%% Back-Cover Texts. A copy of the license is included in LaTeX source
%% format in the file entitled ``fdl.tex''.

\subsubsection{Reserved Symbols}

xqML Symbols with values between 0x02 and 0xFE (both inclusive) are
reserved for grammar specific
purposes. Table~\ref{table:reservedSymbols} lists the used xqML
Symbols in xqML revision \xqMLver{} and their purpose. Entries in
\textit{italics} are productions from the xqML grammar listed in \S
\ref{section:xqml-gram} and the respective production numbers are
provided in brackets.

\begin{table}[!hbp]
\begin{tabular}{|c|c|l|}
\hline
\multicolumn{2}{|c|}{\textbf{Symbol Value}} &
\multicolumn{1}{|c|}{\textbf{Purpose}} \\
\cline{1-2} \textbf{Hex} & \textbf{Dec} & \\ \hline \hline

0x02 & 002 &   \textit{Fmt} (\ref{pr:Fmt})\\ \hline
0x04 & 004 &   \\ \hline
0x06 & 006 &   \textit{Fmt} (\ref{pr:Fmt})\\ \hline
%% 0x08 & 008 &   \\ \hline
%% 0x0A & 010 &   \\ \hline
%% 0x0C & 012 &   \\ \hline
%% 0x0E & 014 &   \\ \hline
%% 0x10 & 016 &   \\ \hline
%% 0x12 & 018 &   \\ \hline
\hline % To graphically signify a break in numbering
0x14 & 020 &   \textit{ATAttribute} (\ref{pr:ATAttribute})\\ \hline
0x16 & 022 &   \textit{ATAttribute} (\ref{pr:ATAttribute})\\ \hline
0x18 & 024 &   \textit{APAttribute} (\ref{pr:APAttribute})\\ \hline
0x1A & 026 &   \textit{APAttribute} (\ref{pr:APAttribute})\\ \hline
0x1C & 028 &   \textit{NSDecl} (\ref{pr:NSDecl})\\ \hline
0x1E & 030 &   Markup flag\\ \hline
0x20 & 032 &   \textit{PI} (\ref{pr:PI})\\ \hline
0x22 & 034 &   \textit{EntityRef} (\ref{pr:EntityRef})\\ \hline
%% 0x24 & 036 &   \textit{EntityRef} (\ref{pr:EntityRef})\\ \hline
0x24 & 036 &   \\ \hline
0x26 & 038 &   \textit{CharRef} (\ref{pr:CharRef})\\ \hline
0x28 & 040 &   \textit{RegId} (\ref{pr:RegId})\\ \hline
0x2A & 042 &   \textit{RegId} (\ref{pr:RegId})\\ \hline
0x2C & 044 &   \textit{doctypedecl} (\ref{pr:doctypedecl})\\ \hline
0x2E & 046 &   \textit{DTDSect} (\ref{pr:DTDSect})\\ \hline
0x30 & 048 &   \textit{ETag} (\ref{pr:ETag})\\ \hline
0x32 & 050 &   \textit{ELFlags} (\ref{pr:ELFlags})\\ \hline
0x34 & 052 &   \textit{ELFlags} (\ref{pr:ELFlags})\\ \hline
0x36 & 054 &   \textit{ELFlags} (\ref{pr:ELFlags})\\ \hline
0x38 & 056 &   \textit{ELFlags} (\ref{pr:ELFlags})\\ \hline
0x3A & 058 &   \textit{ELFlags} (\ref{pr:ELFlags})\\ \hline
0x3C & 060 &   \textit{ELFlags} (\ref{pr:ELFlags})\\ \hline
0x3E & 062 &   \textit{ELFlags} (\ref{pr:ELFlags})\\ \hline
0x40 & 064 &   xqA end marker\\ \hline
%% \hline
%% \multicolumn{3}{|l|}{The remaining symbols are unutilized so far} \\
%% \hline
\end{tabular}
\caption{Table of Reserved Symbols}
\label{table:reservedSymbols}
\end{table}


\subsection{xqML Grammar (Revision \xqMLver{})} \label{section:xqml-gram}
%% Author: Tahir Hashmi
%% 
%% Copyright (C) 2002 2003, Xqueeze Developers
%% 
%% This file is part of Xqueeze Documentation
%% 
%% Permission is granted to copy, distribute and/or modify this
%% document under the terms of the GNU Free Documentation License,
%% Version 1.2 or any later version published by the Free Software
%% Foundation; with no Invariant Sections, no Front-Cover Texts, and no
%% Back-Cover Texts. A copy of the license is included in LaTeX source
%% format in the file entitled ``fdl.tex''.

\subsubsection{Terminals}
\begin{itemize}
\item \textit{Figures enclosed within braces ($\{\}$)} are hex codes for the value
  of an xqML Symbol that should occur within.
\item \textit{Rev} is an octet to be interpreted as an unsigned integer.
\item \textit{xqA} is the inline Xqueeze Association with prolog
  (see \S{} \ref{section:xqa-fmt}).
\item \textit{ELSymbol}, \textit{ATSymbol}, \textit{APSymbol},
  \textit{VASymbol} and \textit{ENSymbol} are all xqML symbols derived
  from an Xqueeze Association to represent the vocabulary of an XML
  document type.
\item \textit{NSSymbol} is an xqML Symbol of the type ``namespace prefix''
  (has a document-specific value).
\item \textit{ElementsToClose} is an octet to be interpreted as an unsigned
  integer.
\item \textit{VUint} is a special type of xqML Symbols that represents
  Variable-length Unsigned integers (see \S{} \ref{section:xqMLSymbols}).
\end{itemize}

\subsubsection{Productions}
\begin{enumerate}
\item \label{pr:document} \(document\;::=\;prolog\;element\;PI* \)

Every xqML document must match the above production. Thus,
\textit{document} is the starting symbol.

\item \label{pr:prolog} \(prolog\;::=\;[^\wedge\{0x1E\}]\!*\;xqMLDecl\;PI\!*\;(doctypedecl\;PI*)? \)

The \textit{prolog} of an xqML document can contain anything upto the
first occurrence of xqML Symbol \{0x1E\}.

\item \label{pr:xqMLDecl} \(xqMLDecl\;::=\;'\{0x1E\}'\;Fmt\;Rev\;Char* \)

Every xqML document must declare what it is (xqML), its binary format
and the version of its encoding. {\it Rev} is an octet that represents
the revision number of the xqML encoding used (see the change in \S{}
\ref{subsubsection:changesRev4}). This octet should be interpreted as
an unsigned integer.

\item \label{pr:Fmt} \(Fmt\;::=\;'\{0x00\}\{0x02\}'\;|\;'\{0x06\}' \)

Format is a sequence that informs the parser whether the stream is
encoded in 8-bit format (like UTF-8) or a 16-bit format encoding (like
UTF-16). Note that this is not entirely dependent on character
encodings since there may be multiple character encodings in each
format. For example, the ISO-8859 family of encodings is 8-bit.

As an example, the xqML counterpart of the XML declaration: 
\begin{center} \verb|<?xml version="1.0" encoding="UTF-8"?>| \end{center}
looks like:
\begin{center} \verb*|    UTF-8| \end{center}
where \verb*| | 
is a visual representation of an xqML Symbol. The symbols in the above
example are 0x1E, 0x00, 0x02, and {\it Rev} -- in that sequence.
Each xqML revision number corresponds to a specific XML version
number. The special attribute ``standalone'' is not written and is
always assumed to be ``no''.

\item \label{pr:doctypedecl} \(doctypedecl\;::=\;('\{0x1E\}\{0x2C\}'\;DoctypeName)\;|\;xqA\;|\;DTDSect \)

An xqML document may declare its document type in one of three ways:
\begin{enumerate}
\item Declare a \textit{DoctypeName} (production
\ref{pr:DoctypeName}) that identifies an external xqA specification
\item Include an xqA specification (including prolog) inline
\item Include a DTD inline in a \textit{DTDSect} (production
\ref{pr:DTDSect})
\end{enumerate}

\item \label{pr:DoctypeName} \(DoctypeName\;::=\;Char* \)

\textit{DoctypeName} should be a valid URI from which an xqA
specification may be retrievable. However, the parser is not
responsible for checking the validity of a \textit{DoctypeName}.

\item \label{pr:DTDSect} \(DTDSect\;::=\;'\{0x1E\}\{0x2E\}'\;Char* \)

\textit{DTDSect} contains an internal DTD in the format specified in
XML 1.0 specification, including the \texttt{DOCTYPE} tag. An xqML
parser must be capable of generating an xqA specification out of the
DTD but is not always required to do so.

\item \label{pr:element} \(element\;::=\;NSDecl\!*\;RegId\!*\;STag\;(content\;ETag?)? \)

This corresponds to an XML Element. The element must have a start tag
\textit{STag}. The start tag also contains an indication of whether
the element is empty or not. If the element is not empty, it would
also contain \textit{content} and a closing tag.  The closing tag
\textit{ETag} is optional since several consecutive closing tags are
combined into one in xqML.

\item \label{pr:NSDecl} \(NSDecl\;::=\;'\{0x1E\}\{0x1C\}'\;Char*\;'\{0x1E\}'\;Char* \)

These are the xqML equivalents of xmlns declarations in XML. For
example, the declaration: \\
\verb|xmlns:xsl="http://www.w3.org/1999/XSL/Transform"| \\
would be encoded as:
\begin{center} \verb*|  xsl http://www.w3.org/1999/XSL/Transform| \end{center}
There may be a null string instead of ``xsl'' in the above example.

\item \label{pr:RegId} \(RegId\;::=\;'\{0x1E\}'\;('\{0x2A\}'\;|\;('\{0x28\}'\;NSSymbol))\;Char\!* \)

This production corresponds to an identifier registration in the
Dynamic Association mapping of the document (see
\S{}\ref{section:DynamicAssoc}). The string at the end of this
production is taken as the identifier to be registered. The
declaration may explicitly indicate association with a particular
namespace through the use of an \textit{NSSymbol}.

\item \label{pr:STag} \(STag\;::=\;'\{0x1E\}'\;(ELFlags\;NSSymbol?)?\;ELSymbol\;attribute\!* \)

This represents an element start tag. \textit{ELFlags} is an octet
that has three status flag bits. \textit{NSSymbol} is a symbol for XML
Namespace prefix. ELSymbol is the symbol for the element's
identifier. This may be followed by any number of attributes or XML
Namespace declarations (\textit{NSDecl}).

\item \label{pr:ELFlags} \(ELFlags\;::=\; 0x32 - 0x3E \)

This octet contains three status flags in its 2$^{nd}$, 3$^{rd}$ and
4$^{th}$ least significant bits  to signify the following:
\begin{enumerate}
\item {\it Empty Element}: The second least significant bit of the
  octet is set if the element is empty
\item {\it Namespace Prefix}: The third least significant bit is set
  if an {\it NSSymbol} follows
\item {\it Close Previous}: If the fourth least significant bit is
  set, it indicates that the last open element should be closed.
\end{enumerate}
The four most significant bits are \verb|0011|. Therefore this octet
can have values between 0x30 and 0x3E. However, if all the flag bits
are unset, the resultant value, 0x30, is never written. This value is
used to indicate one or more closing tags (See production
\ref{pr:ETag}).

\item \label{pr:attribute} \(attribute\;::=\;ATAttribute\;|\;APAttribute \)

Attributes may have unspecified values (\textit{ATAttribute}) or
values that have been assigned symbols in the xqA specification
(\textit{APAttribute}).

\item \label{pr:ATAttribute} \(ATAttribute\;::=\;(('\{0x14\}'\;NSSymbol)\;|\;'{0x16}')\;ATSymbol\;Char*\;(Reference\;Char*)*\;'\{0x16\}' \)

An attribute is started by the symbol 0x16, or by the symbol 0x14
followed by an {\it NSSymbol}. The symbol for the attribute
identifier, {\it ATSymbol}, comes next. The attribute is closed by the
symbol 0x16. Any character data or references before the closing
delimiter is taken to be the value of the attribute.

\item \label{pr:APAttribute} \(APAttribute\;::=\;(('\{0x18\}'\;NSSymbol)\;|\;'{0x1A}')\;APSymbol\;VASymbol \)

Attributes with predefined values begin with the symbol 0x18, or by
the symbol 0x1A followed by an {\it NSSymbol}. \textit{APSymbol} is
the symbol for the attribute identifier and \textit{VASymbol} is the
symbol for it's value. These attributes are completely represented by
symbols.

For example, the xqML counterpart of \\
\verb|<ufn:file path="/etc/issue.net" binary="no"/>|, where the
attribute ``binary'' has enumerated values ``yes'' and ``no'',  would
be:
\begin{center} \verb*|      /etc/issue.net    | \end{center}
Here we have six symbols, followed by the string ``/etc/issue.net''
followed by four more symbols. The symbols would be:
\begin{enumerate}
\item 0x1E
\item 0x36 (ELFlags, indicating an empty element and a namespace
  prefix to follow)
\item A document specific symbol for the namespace prefix ``ufn''
\item The symbol for element identifier ``file''
\item 0x16 -- to signify an attribute of type \textit{ATAttribute}
\item The symbol for attribute identifier ``path''
\end{enumerate}
The value of ``path'' follows as char data. The next four symbols
would be:
\begin{enumerate}
\item 0x16 -- to mark the end of attribute ``path''
\item 0x1A -- to signify an attribute of type \textit{APAttribute}
\item The symbol for attribute identifier ``binary''
\item The symbol for attribute value ``no''
\end{enumerate}

\item \label{pr:content} \(content\;::=\;Char*\;((element\;|\;Reference\;|\;PI)\;Char*)* \)

An element may contain character data and any number of child
elements, references or character data in any order. Restrictions
imposed by document type specifications (DTD, XML Schema etc.) may
apply while validating.

\item \label{pr:Reference} \(Reference\;::=\;EntityRef\;|\;CharRef \)

\item \label{pr:EntityRef} \(EntityRef\;::=\;'\{0x22\}'\;ENSymbol \)

This production matches an entity reference. ENSymbol is the symbol
for the entity identifier, \emph{not} its expansion.

\item \label{pr:CharRef} \(CharRef\;::=\;'\{0x1E\}\{0x26\}'\;VUint\; \)

This production matches a Character Reference. {\it VUint} is a
Variable-length Unsigned integer, whose value equals the code point of
the desired character.

\item \label{pr:ETag} \(ETag\;::=\;'\{0x1E\}\{0x30\}'\;ElementsToClose \)

The closing tag has an octet \textit{ElementsToClose} which should be
interpreted as the binary representation of an unsigned integer, whose
value signifies the number of elements to close in correct (stack)
order.

\item \label{pr:PI} \(PI\;::=\;'\{0x1E\}\{0x20\}'\;PITarget\;'\{0x1E\}'\;PIContent\;'\{0x1E\}' \)

This is a representation of an XML Processing
Instruction. \textit{PITarget} is the equivalent of targets in XML
PIs. \textit{PIContent} is the data that is passed on to the
application. For example, a hypothetical SSI include directive for a
web server may be written in XML as
\verb|<?ssi includefile("headers.shtml")?>|. The xqML equivalent of
this would be:
\begin{center} \verb*|  ssi includefile("headers.shtml") | \end{center}
where the symbols are 0x1E, 0x02, 0x1E and 0x1E in that order.

\item \label{pr:PITarget} \(PITarget\;::=\;Char* \) 

\item \label{pr:PIContent} \(PIContent\;::=\;Char* \)

\item \label{pr:Char} \(Char\;::=\;0x09\;|\;0x0A\;|\;0x0D\;|\;[0x20-0xD7FF]\;|\;[0xE00-0xFFFD]\;|\;[0x10000-0x10FFFF] \)

xqML characters are exactly same as XML characters. Additionaly, the
characters `$<$', `$>$', `\verb|'|', `\verb|"|' and `\&' need not be
escaped, unlike XML.

\end{enumerate}


\section{Xqueeze Association}

Xqueeze uses an association between symbols and their corresponding
XML identifiers and types as defined in a specification
(DTD/Schema). This enables representation of known identifiers in the
markup with symbols. Associating the type of an identifier along with
it's name also makes it easy to various structural units of the
document without having to use too many special characters and
character-combinations.

\subsection{Xqueeze Association Algorithm (Version \xqAalgo{})} \label{section:xqa-algo}
%% Author: Tahir Hashmi
%% 
%% Copyright (C) 2002 2003, Xqueeze Developers
%% 
%% This file is part of Xqueeze Documentation
%% 
%% Permission is granted to copy, distribute and/or modify this
%% document under the terms of the GNU Free Documentation License,
%% Version 1.2 or any later version published by the Free Software
%% Foundation; with no Invariant Sections, no Front-Cover Texts, and no
%% Back-Cover Texts. A copy of the license is included in LaTeX source
%% format in the file entitled ``fdl.tex''.

This is the allgorithm that is used to map the identifiers found in a
DTD/Schema to xqML Symbols. The steps of the algorithm are:
\begin{enumerate}
\item collect all Element identifiers
\item collect all Attribute identifiers
\item collect all Enumerated Attribute Value identifiers
\item collect all Entity References together
\item merge the above collections, discarding duplicates
\item sort the merged collection lexically on the values of unicode
  code-points
\item assign symbols starting from 256 in ascending order to the
  identifiers
\end{enumerate}

This simple algorithm assures that the assignments would remain the
same even if a particular specification (DTD/Schema) has slight
variations in the way it's written in the generator's and consumer's
copies, as long as both define the same vocabulary. Note that it is
not dependent on the structure of the document.

\subsubsection{Dynamic Associations} \label{section:DynamicAssoc}

Xqueeze allows for associating symbols to identifiers within a running
document through {\it Dynamic Associations}. This allows for
generation of xqML documents without the knowledge of whole or part of
the schema. Dynamic Associations cover elements, attributes and entity
references. Attribute values are not covered, and should be written as
string literals.

For assigning symbols to dynamically declared identifiers, the
processor must maintain a separate lookup table for each namespace
with which one or more dynamic identifier declarations are
associated. The namespace with which to associate a dynamically
declared identifier is determined by these rules:
\begin{enumerate}
\item Declarations appearing ahead of an element are associated with
  the namespace that the element is associated with
\item Declarations with explicit namespace prefixes are associated
  with the namespace denoted by the prefix, provided the prefix is
  valid and legal
\end{enumerate}

While registering identifiers dynamically, duplicate declarations
within the same namespace are discarded. This means that identifiers that
already exist in a given namespace would not be
re-assigned.\footnote{Therefore it is a good practice to declare
  dynamic identifiers in separate namespace(s) while mixing with
  various vocabularies.} Symbols are assigned to identifiers in the
order of their appearance in the document, starting from the first
unused symbol in the Association corresponding to the namespace in
context.

Portability of such associations is limited to the document that
contained the declarations and parts of the document using dynamically
assigned symbols can't be used elsewhere, without translation and
re-assigning of symbols. Nor can the document be safely modified
without preserving the declarations.


\subsection{Xqueeze Association Format (Version \xqAfmt{})} \label{section:xqa-fmt}
%% Author: Tahir Hashmi
%% 
%% Copyright (C) 2002 - 2004, Xqueeze Developers
%% 
%% This file is part of Xqueeze Documentation
%% 
%% Permission is granted to copy, distribute and/or modify this
%% document under the terms of the GNU Free Documentation License,
%% Version 1.2 or any later version published by the Free Software
%% Foundation; with no Invariant Sections, no Front-Cover Texts, and no
%% Back-Cover Texts. A copy of the license is included in LaTeX source
%% format in the file entitled ``fdl.tex''.

Xqueeze associations are represented in a format that itself is quite
compact and uses xqML Symbols themselves. The specification begins
with an optional prolog whose format resembles that of an xqML
\textit{PI} (Processing Instruction): \\
\('\{0x1E\}\{0x20\}xqa\{0x1E\}'\;Char*\;'\{0x1E\}' \)

Here, \textit{Char*} may contain the identification string for the
document type. The prolog is followed by individual entries for
identifiers.

Individual entries are listed as `\{0x1E\}', followed by a symbol,
followed by a string that the symbol represents. The end of
specifications is denoted by the sequence `\{0x1E\}\{0x40\}'. This
structure enables inline specification of the symbols associations, if
required by a document.


\section{Changes}

\subsection{xqML}
\subsubsection{Revision 4} \label{subsubsection:changesRev4}
\begin{itemize}
\item xqML will now have ``Revisions'' instead of version numbers. The
  current format can report a maximum of 255 revisions. However, this
  does not imply that there will not be more than 255 revisions of
  xqML
\item Comments have now been dropped
\item CDATA Sections have now been dropped
\item The format now allows for generation of documents without prior
  knowledge of schema through {\it Dynamic Associations} (\S{}
  \ref{pr:RegId})
\item The {\it xqMLDecl} represents xqML revision information in
  binary now
\item A new terminal {\it Rev} has been added
\item {\it xqMLDecl} is now mandatory for all xqML documents
\item All xqML documents have the value of special attribute
  ``standalone'' as ``no''
\item {\it ELFlags} production added to combine three flags related to
  elements into one octet
\item {\it ATAttribute} ends with `\{0x16\}' instead of
  `\{0x1E\}\{0x16\}'
\item A new terminal and xqML Symbol type, {\it VUint} has been added
  (see \S{} \ref{section:xqMLSymbols})
\item {\it CharRef} now uses VUint to encode the character's code
  point value
\item The production {\it EE\_STag} has been dropped
\item {\it ETag} now uses `\{0x30\}' instead of `\{0x3E\}'.
\end{itemize}
\subsubsection{Version 0.3}
\begin{itemize}
\item Anything is permissible upto the occurrence of \textit{xqMLDecl}
  in a document
\item A new production, \textit{PI}, has been added for Processing
  Instructions
\item \textit{doctypedecl} now starts with `\{0x1E\}\{0x2C\}' instead
  of `\{0x1E\}\{0x12\}'
\item \textit{xqA} should necessarily include a prolog now
\item \textit{doctypedecl} may now have an inline DTD with a new
  production \textit{DTDSect}.
\item \textit{element} production was erroneous till the last
  version
\item A new prodcution \textit{NSPrefix} has been added for XML
  Namespace prefixes
\item The productions \textit{EE\_STag}, \textit{STag},
  \textit{ATAttribute}, \textit{APAttribute} and \textit{EntityRef}
  can now have namespace prefixes
\item \textit{EE\_STag} starts with `\{0x1E\}\{0x2A\}' instead of
  `\{0x1E\}'
\item \textit{ATAttribute} starts with `\{0x16\}' instead of
  `\{0x1E\}'
\item \textit{APAttribute} starts with `\{0x18\}' instead of
  `\{0x1E\}'
\item \textit{EntityRef} starts with `\{0x1E\}\{0x24\}' instead of
  `\{0x1E\}'
\item \textit{CDSect} starts with `\{0x1E\}\{0x28\}' instead of
  \textit{CDDelim} and ends with `\{0x1E\}' instead of
  \textit{CDDelim} (\{0x1E\}\{0x14\}).
\item \textit{Char} now matches the \textit{Char} production in XML
  1.0 grammar specification.
\end{itemize}
\subsubsection{Version 0.2}
\begin{itemize}
\item xqML Symbol `\{0x1E\}' replaces `$<$' for the latter's role in
  xqML markup
\item \textit{Attribute} is split into \textit{ATAttribute} and
  \textit{APAttribute}, together referred as \textit{attribute}.
\item \textit{ATAttribute} can contain \textit{Reference}.
\item \textit{ATAttribute} is terminated by `\{0x1E\}\{0x16\}' instead of
  `$<$'
\item \textit{CharRef} starts with `\{0x1E\}\{0x26\}' instead of
  `\&\{0x26\}'
\item \textit{CharRef} ends with `\{0x1E\}'. Earlier there was no
  end-marker
\item \textit{Comment} ends with `\{0x1E\}' instead of \textit{ETag?}
\item \textit{Comment} is deprecated
\item \textit{Char} is a terminal that matches any printable character
\item \textit{Num} does not contain `.'
\end{itemize}
\subsubsection{Version 0.1}
First Release

\subsection{Xqueeze Association algorithm}
\subsubsection{Version 0.2}
\begin{itemize}
\item Removed distinction of identifiers based on type
\item Added support for Dynamic Associations
\end{itemize}
\subsubsection{Version 0.1}
First Release

\subsection{Xqueeze Association format}
\subsubsection{Version 0.3}
\begin{itemize}
\item Removed section markers
\item xqA specification now ends with the sequence `\{0x1E\}\{0x40\}'
  instead of `\{0x1E\}\{0x3C\}'.
\end{itemize}
\subsubsection{Version 0.2}
\begin{itemize}
\item xqML Symbol `\{0x1E\}' replaces `$<$' for the latter's role in
  xqA format.
\item The prolog format has been changed to resemble an xqML PI.
\item Reserved symbols used in the previous version have been shifted
  44 decimal values up. For example, the symbol for Element section is
  now `\{0x30\}' (48) instead of `\{0x04\}' (04).
\end{itemize}
\subsubsection{Version 0.1}
First Release

\section{GNU Free Documentation License} \label{section:gfdl}
\begin{scriptsize}
\begin{center}
		GNU Free Documentation License \\
		  Version 1.2, November 2002
\end{center}

 Copyright \copyright 2000,2001,2002  Free Software Foundation, Inc. \\
     59 Temple Place, Suite 330, Boston, MA  02111-1307  USA \\
 Everyone is permitted to copy and distribute verbatim copies
 of this license document, but changing it is not allowed.


\subsection*{Preamble}

The purpose of this License is to make a manual, textbook, or other
functional and useful document "free" in the sense of freedom: to
assure everyone the effective freedom to copy and redistribute it,
with or without modifying it, either commercially or noncommercially.
Secondarily, this License preserves for the author and publisher a way
to get credit for their work, while not being considered responsible
for modifications made by others.

This License is a kind of "copyleft", which means that derivative
works of the document must themselves be free in the same sense.  It
complements the GNU General Public License, which is a copyleft
license designed for free software.

We have designed this License in order to use it for manuals for free
software, because free software needs free documentation: a free
program should come with manuals providing the same freedoms that the
software does.  But this License is not limited to software manuals;
it can be used for any textual work, regardless of subject matter or
whether it is published as a printed book.  We recommend this License
principally for works whose purpose is instruction or reference.


\subsection{Applicability and Definitions}

This License applies to any manual or other work, in any medium, that
contains a notice placed by the copyright holder saying it can be
distributed under the terms of this License.  Such a notice grants a
world-wide, royalty-free license, unlimited in duration, to use that
work under the conditions stated herein.  The "Document", below,
refers to any such manual or work.  Any member of the public is a
licensee, and is addressed as "you".  You accept the license if you
copy, modify or distribute the work in a way requiring permission
under copyright law.

A "Modified Version" of the Document means any work containing the
Document or a portion of it, either copied verbatim, or with
modifications and/or translated into another language.

A "Secondary Section" is a named appendix or a front-matter section of
the Document that deals exclusively with the relationship of the
publishers or authors of the Document to the Document's overall subject
(or to related matters) and contains nothing that could fall directly
within that overall subject.  (Thus, if the Document is in part a
textbook of mathematics, a Secondary Section may not explain any
mathematics.)  The relationship could be a matter of historical
connection with the subject or with related matters, or of legal,
commercial, philosophical, ethical or political position regarding
them.

The "Invariant Sections" are certain Secondary Sections whose titles
are designated, as being those of Invariant Sections, in the notice
that says that the Document is released under this License.  If a
section does not fit the above definition of Secondary then it is not
allowed to be designated as Invariant.  The Document may contain zero
Invariant Sections.  If the Document does not identify any Invariant
Sections then there are none.

The "Cover Texts" are certain short passages of text that are listed,
as Front-Cover Texts or Back-Cover Texts, in the notice that says that
the Document is released under this License.  A Front-Cover Text may
be at most 5 words, and a Back-Cover Text may be at most 25 words.

A "Transparent" copy of the Document means a machine-readable copy,
represented in a format whose specification is available to the
general public, that is suitable for revising the document
straightforwardly with generic text editors or (for images composed of
pixels) generic paint programs or (for drawings) some widely available
drawing editor, and that is suitable for input to text formatters or
for automatic translation to a variety of formats suitable for input
to text formatters.  A copy made in an otherwise Transparent file
format whose markup, or absence of markup, has been arranged to thwart
or discourage subsequent modification by readers is not Transparent.
An image format is not Transparent if used for any substantial amount
of text.  A copy that is not "Transparent" is called "Opaque".

Examples of suitable formats for Transparent copies include plain
ASCII without markup, Texinfo input format, LaTeX input format, SGML
or XML using a publicly available DTD, and standard-conforming simple
HTML, PostScript or PDF designed for human modification.  Examples of
transparent image formats include PNG, XCF and JPG.  Opaque formats
include proprietary formats that can be read and edited only by
proprietary word processors, SGML or XML for which the DTD and/or
processing tools are not generally available, and the
machine-generated HTML, PostScript or PDF produced by some word
processors for output purposes only.

The "Title Page" means, for a printed book, the title page itself,
plus such following pages as are needed to hold, legibly, the material
this License requires to appear in the title page.  For works in
formats which do not have any title page as such, "Title Page" means
the text near the most prominent appearance of the work's title,
preceding the beginning of the body of the text.

A section "Entitled XYZ" means a named subunit of the Document whose
title either is precisely XYZ or contains XYZ in parentheses following
text that translates XYZ in another language.  (Here XYZ stands for a
specific section name mentioned below, such as "Acknowledgements",
"Dedications", "Endorsements", or "History".)  To "Preserve the Title"
of such a section when you modify the Document means that it remains a
section "Entitled XYZ" according to this definition.

The Document may include Warranty Disclaimers next to the notice which
states that this License applies to the Document.  These Warranty
Disclaimers are considered to be included by reference in this
License, but only as regards disclaiming warranties: any other
implication that these Warranty Disclaimers may have is void and has
no effect on the meaning of this License.


\subsection{Verbatim Copying}

You may copy and distribute the Document in any medium, either
commercially or noncommercially, provided that this License, the
copyright notices, and the license notice saying this License applies
to the Document are reproduced in all copies, and that you add no other
conditions whatsoever to those of this License.  You may not use
technical measures to obstruct or control the reading or further
copying of the copies you make or distribute.  However, you may accept
compensation in exchange for copies.  If you distribute a large enough
number of copies you must also follow the conditions in section 3.

You may also lend copies, under the same conditions stated above, and
you may publicly display copies.


\subsection{Copying in Quantity}

If you publish printed copies (or copies in media that commonly have
printed covers) of the Document, numbering more than 100, and the
Document's license notice requires Cover Texts, you must enclose the
copies in covers that carry, clearly and legibly, all these Cover
Texts: Front-Cover Texts on the front cover, and Back-Cover Texts on
the back cover.  Both covers must also clearly and legibly identify
you as the publisher of these copies.  The front cover must present
the full title with all words of the title equally prominent and
visible.  You may add other material on the covers in addition.
Copying with changes limited to the covers, as long as they preserve
the title of the Document and satisfy these conditions, can be treated
as verbatim copying in other respects.

If the required texts for either cover are too voluminous to fit
legibly, you should put the first ones listed (as many as fit
reasonably) on the actual cover, and continue the rest onto adjacent
pages.

If you publish or distribute Opaque copies of the Document numbering
more than 100, you must either include a machine-readable Transparent
copy along with each Opaque copy, or state in or with each Opaque copy
a computer-network location from which the general network-using
public has access to download using public-standard network protocols
a complete Transparent copy of the Document, free of added material.
If you use the latter option, you must take reasonably prudent steps,
when you begin distribution of Opaque copies in quantity, to ensure
that this Transparent copy will remain thus accessible at the stated
location until at least one year after the last time you distribute an
Opaque copy (directly or through your agents or retailers) of that
edition to the public.

It is requested, but not required, that you contact the authors of the
Document well before redistributing any large number of copies, to give
them a chance to provide you with an updated version of the Document.


\subsection{Modifications}

You may copy and distribute a Modified Version of the Document under
the conditions of sections 2 and 3 above, provided that you release
the Modified Version under precisely this License, with the Modified
Version filling the role of the Document, thus licensing distribution
and modification of the Modified Version to whoever possesses a copy
of it.  In addition, you must do these things in the Modified Version:

\noindent A. Use in the Title Page (and on the covers, if any) a title distinct
   from that of the Document, and from those of previous versions
   (which should, if there were any, be listed in the History section
   of the Document).  You may use the same title as a previous version
   if the original publisher of that version gives permission.

\noindent B. List on the Title Page, as authors, one or more persons or entities
   responsible for authorship of the modifications in the Modified
   Version, together with at least five of the principal authors of the
   Document (all of its principal authors, if it has fewer than five),
   unless they release you from this requirement.

\noindent C. State on the Title page the name of the publisher of the
   Modified Version, as the publisher.

\noindent D. Preserve all the copyright notices of the Document.

\noindent E. Add an appropriate copyright notice for your modifications
   adjacent to the other copyright notices.

\noindent F. Include, immediately after the copyright notices, a license notice
   giving the public permission to use the Modified Version under the
   terms of this License, in the form shown in the Addendum below.

\noindent G. Preserve in that license notice the full lists of Invariant Sections
   and required Cover Texts given in the Document's license notice.

\noindent H. Include an unaltered copy of this License.

\noindent I. Preserve the section Entitled "History", Preserve its Title, and add
   to it an item stating at least the title, year, new authors, and
   publisher of the Modified Version as given on the Title Page.  If
   there is no section Entitled "History" in the Document, create one
   stating the title, year, authors, and publisher of the Document as
   given on its Title Page, then add an item describing the Modified
   Version as stated in the previous sentence.

\noindent J. Preserve the network location, if any, given in the Document for
   public access to a Transparent copy of the Document, and likewise
   the network locations given in the Document for previous versions
   it was based on.  These may be placed in the "History" section.
   You may omit a network location for a work that was published at
   least four years before the Document itself, or if the original
   publisher of the version it refers to gives permission.

\noindent K. For any section Entitled "Acknowledgements" or "Dedications",
   Preserve the Title of the section, and preserve in the section all
   the substance and tone of each of the contributor acknowledgements
   and/or dedications given therein.

\noindent L. Preserve all the Invariant Sections of the Document,
   unaltered in their text and in their titles.  Section numbers
   or the equivalent are not considered part of the section titles.

\noindent M. Delete any section Entitled "Endorsements".  Such a section
   may not be included in the Modified Version.

\noindent N. Do not retitle any existing section to be Entitled "Endorsements"
   or to conflict in title with any Invariant Section.
\noindent O. Preserve any Warranty Disclaimers.

If the Modified Version includes new front-matter sections or
appendices that qualify as Secondary Sections and contain no material
copied from the Document, you may at your option designate some or all
of these sections as invariant.  To do this, add their titles to the
list of Invariant Sections in the Modified Version's license notice.
These titles must be distinct from any other section titles.

You may add a section Entitled "Endorsements", provided it contains
nothing but endorsements of your Modified Version by various
parties--for example, statements of peer review or that the text has
been approved by an organization as the authoritative definition of a
standard.

You may add a passage of up to five words as a Front-Cover Text, and a
passage of up to 25 words as a Back-Cover Text, to the end of the list
of Cover Texts in the Modified Version.  Only one passage of
Front-Cover Text and one of Back-Cover Text may be added by (or
through arrangements made by) any one entity.  If the Document already
includes a cover text for the same cover, previously added by you or
by arrangement made by the same entity you are acting on behalf of,
you may not add another; but you may replace the old one, on explicit
permission from the previous publisher that added the old one.

The author(s) and publisher(s) of the Document do not by this License
give permission to use their names for publicity for or to assert or
imply endorsement of any Modified Version.


\subsection{Combining Documents}

You may combine the Document with other documents released under this
License, under the terms defined in section 4 above for modified
versions, provided that you include in the combination all of the
Invariant Sections of all of the original documents, unmodified, and
list them all as Invariant Sections of your combined work in its
license notice, and that you preserve all their Warranty Disclaimers.

The combined work need only contain one copy of this License, and
multiple identical Invariant Sections may be replaced with a single
copy.  If there are multiple Invariant Sections with the same name but
different contents, make the title of each such section unique by
adding at the end of it, in parentheses, the name of the original
author or publisher of that section if known, or else a unique number.
Make the same adjustment to the section titles in the list of
Invariant Sections in the license notice of the combined work.

In the combination, you must combine any sections Entitled "History"
in the various original documents, forming one section Entitled
"History"; likewise combine any sections Entitled "Acknowledgements",
and any sections Entitled "Dedications".  You must delete all sections
Entitled "Endorsements".


\subsection{Collections of Documents}

You may make a collection consisting of the Document and other documents
released under this License, and replace the individual copies of this
License in the various documents with a single copy that is included in
the collection, provided that you follow the rules of this License for
verbatim copying of each of the documents in all other respects.

You may extract a single document from such a collection, and distribute
it individually under this License, provided you insert a copy of this
License into the extracted document, and follow this License in all
other respects regarding verbatim copying of that document.


\subsection{Aggregation with Independent Works}

A compilation of the Document or its derivatives with other separate
and independent documents or works, in or on a volume of a storage or
distribution medium, is called an "aggregate" if the copyright
resulting from the compilation is not used to limit the legal rights
of the compilation's users beyond what the individual works permit.
When the Document is included an aggregate, this License does not
apply to the other works in the aggregate which are not themselves
derivative works of the Document.

If the Cover Text requirement of section 3 is applicable to these
copies of the Document, then if the Document is less than one half of
the entire aggregate, the Document's Cover Texts may be placed on
covers that bracket the Document within the aggregate, or the
electronic equivalent of covers if the Document is in electronic form.
Otherwise they must appear on printed covers that bracket the whole
aggregate.


\subsection{Translation}

Translation is considered a kind of modification, so you may
distribute translations of the Document under the terms of section 4.
Replacing Invariant Sections with translations requires special
permission from their copyright holders, but you may include
translations of some or all Invariant Sections in addition to the
original versions of these Invariant Sections.  You may include a
translation of this License, and all the license notices in the
Document, and any Warrany Disclaimers, provided that you also include
the original English version of this License and the original versions
of those notices and disclaimers.  In case of a disagreement between
the translation and the original version of this License or a notice
or disclaimer, the original version will prevail.

If a section in the Document is Entitled "Acknowledgements",
"Dedications", or "History", the requirement (section 4) to Preserve
its Title (section 1) will typically require changing the actual
title.


\subsection{Termination}

You may not copy, modify, sublicense, or distribute the Document except
as expressly provided for under this License.  Any other attempt to
copy, modify, sublicense or distribute the Document is void, and will
automatically terminate your rights under this License.  However,
parties who have received copies, or rights, from you under this
License will not have their licenses terminated so long as such
parties remain in full compliance.


\subsection{Future Revisions of This License}

The Free Software Foundation may publish new, revised versions
of the GNU Free Documentation License from time to time.  Such new
versions will be similar in spirit to the present version, but may
differ in detail to address new problems or concerns.  See
http://www.gnu.org/copyleft/.

Each version of the License is given a distinguishing version number.
If the Document specifies that a particular numbered version of this
License "or any later version" applies to it, you have the option of
following the terms and conditions either of that specified version or
of any later version that has been published (not as a draft) by the
Free Software Foundation.  If the Document does not specify a version
number of this License, you may choose any version ever published (not
as a draft) by the Free Software Foundation.


\subsection*{ADDENDUM: How to use this License for your documents}

To use this License in a document you have written, include a copy of
the License in the document and put the following copyright and
license notices just after the title page:

\noindent Copyright (c)  YEAR  YOUR NAME. \\
    Permission is granted to copy, distribute and/or modify this document
    under the terms of the GNU Free Documentation License, Version 1.2
    or any later version published by the Free Software Foundation;
    with no Invariant Sections, no Front-Cover Texts, and no Back-Cover Texts.
    A copy of the license is included in the section entitled "GNU
    Free Documentation License".

If you have Invariant Sections, Front-Cover Texts and Back-Cover Texts,
replace the "with...Texts." line with this:

\emph{with the Invariant Sections being LIST THEIR TITLES, with the
    Front-Cover Texts being LIST, and with the Back-Cover Texts being LIST.}

If you have Invariant Sections without Cover Texts, or some other
combination of the three, merge those two alternatives to suit the
situation.

If your document contains nontrivial examples of program code, we
recommend releasing these examples in parallel under your choice of
free software license, such as the GNU General Public License,
to permit their use in free software.

\end{scriptsize}

\end{document}
